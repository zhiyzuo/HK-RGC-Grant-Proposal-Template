\documentclass[12pt,notitlepage]{article}
%%%%%%%%%%%%%%%%%%%%%%%%%
\usepackage{blindtext}
%%%%%%%%%%%%%%%%%%%%%%%%%
\usepackage{fontspec}
\setmainfont{Times New Roman}
%%%%%%%%%%%%%%%%%%%%%%%%%
\usepackage[a4paper,margin=2.5cm]{geometry}
%%%%%%%%%%%%%%%%%%%%%%%%%
\usepackage{setspace}
\singlespacing
%%%%%%%%%%%%%%%%%%%%%%%%%
\usepackage{minted}
%%%%%%%%%%%%%%%%%%%%%%%%%
\usepackage{titling}
%\pretitle{\large\bfseries}
%\posttitle{\vskip 0pt}
%\posttitle{}
%\preauthor{}
%\postauthor{}
\predate{}
\postdate{}
\renewcommand\maketitlehooka{\null\mbox{}\vfill}
\renewcommand\maketitlehookd{\vfill\null}
%%%%%%%%%%%%%%%%%%%%%%%%%
\usepackage[perpage,symbol*]{footmisc}
%\renewcommand{\thefootnote}{\alph{footnote}}
%\newcommand{\numfootnote}{\fnsymbol{footnote}}
%%%%%%%%%%%%%%%%%%%%%%%%%
\usepackage{booktabs}
\usepackage{multirow}
\usepackage{enumitem}
\setenumerate{nolistsep} % kills all vertical spacing
%%%%%%%%%%%%%%%%%%%%%%%%%
\newcommand{\needcite}{\textcolor{red}{[CITE]}}
%%%%%%%%%%%%%%%%%%%%%%%%%
%\title{\vskip -10ex Background of Research}
\title{\vskip -5ex \large \textbf{Something Interesting and Exciting}}
\date{\vskip -3ex}
\author{Zhiya Zuo\\\textit{Prepared for Early-Career Scheme 21/22}}
%%%%%%%%%%%%%%%%%%%%%%%%%
\usepackage{titlesec}
\titlespacing{\title}{0pt}{0.2ex}{0ex}
\titleformat{\title}
{\centering\normalfont\bfseries}
{}{0pt}{}
%
\titlespacing{\section}{0pt}{0.2ex}{0ex}
\titleformat{\section}
{\centering\normalfont\bfseries}
{}{0pt}{}
%
\titlespacing{\subsection}{0pt}{0.2ex}{0ex}
\titleformat{\subsection}
{\normalfont\bfseries\itshape}
{}{0pt}{}
%
\titlespacing{\subsubsection}{0pt}{0.2ex}{0ex}
\titleformat{\subsubsection}
{\normalfont}
{}{0pt}{}
%%%%%%%%%%%%%%%%%%%%%%%%%
\usepackage{lipsum}
%%%%%%%%%%%%%%%%%%%%%%%%%
\usepackage{float}
\usepackage{graphicx}
\usepackage{rotating}
\usepackage{amsmath}
\usepackage{hyperref}
\usepackage[capitalise,noabbrev]{cleveref}
%%%%%%%%%%%%%%%%%%%%%%%%%
%\pagenumbering{gobble}
%%%%%%%%%%%%%%%%%%%%%%%%%
\usepackage{csquotes}
%%%%%%%%%%%%%%%%%%%%%%%%%
\usepackage[american]{babel}
\usepackage[sortcites=true,
			sorting=nyt,
			style=apa,
			doi=false,
			isbn=false,
			maxcitenames=3,
			mincitenames=1,
			maxbibnames=3,
			minbibnames=1,
			backend=biber,
			eprint=false]{biblatex}
\DeclareLanguageMapping{american}{american-apa}
\DeclareDelimFormat[textcite]{finalnamedelim}{%
  \ifnumgreater{\value{liststop}}{2}{\finalandcomma}{}%
  \addspace\bibstring{and}\space}
%\DeclareDelimFormat[parencite]{finalnamedelim}{\addspace\&\space}
\addbibresource{references.bib}
%%%%%%%%%%%%%%%%%%%%%%%%%

\begin{document}

\begin{titlingpage}
    \maketitle
    \noindent
\normalsize
\textbf{Keywords:}
RGC,
UGC,
ECS,
GRF,
\LaTeX~template
\end{titlingpage}
\pagenumbering{gobble}


\clearpage
\section{Abstract}
\smallskip

This is an unofficial \LaTeX~template based on RGC's explanatory note.
It may change from time to time. Use at your own risk.

Use \textbackslash medskip to increase spacing.
\blindtext

\medskip

\blindtext
%%%%%%%%%%%%%%%%%%%%%%%%%
\clearpage
\section{Project Objectives}
\smallskip

\begin{enumerate}
    \itemsep1em
    \item \blindtext[1]
    %%%%%%%%%%%%%%%
    \item \blindtext[1]
\end{enumerate}

%%%%%%%%%%%%%%%%%%%%%%%%%
\clearpage
\section{Pathways to Impact Statement}
\smallskip

\blindtext[1]

\smallskip

\blindtext[2]

\smallskip
%%%%%%%%%%%%%%%%%%%%%%%%%
\clearpage
\section{Education Plan}
\smallskip

\blindtext[3]
%%%%%%%%%%%%%%%%%%%%%%%%%
\clearpage
\pagenumbering{arabic}
\setcounter{page}{1}

%Thanks to the information and communication technologies (ICTs),
%Along with the increasingly specialized knowledge body,
%it is more difficult, if not impossible,
%for individuals to possess all expected skills %and abilities due to the growing task complexity and competitiveness.
%Organizations, as a result, have become increasingly relying on teams
%\footnote{While some scholars consider members of a team hold specific roles while those of a group do not, such subtle difference is of little importance in this proposal. Following previous studies~\parencite[e.g.,][]{webber2001impact,mathieu2017century,park2020understanding}, we use teams and groups interchangeably.} 
%as their basic operation units~\parencite{van2016past,mathieu2019embracing}.

%Indeed, crowdsourcing contests have shown great successes across various areas. In May 2011, NASA, European Space Agency, and Royal Astronomical Society cohosted an open challenge on mapping dark matter, a great unifying problem of the universe we are living in over the past decades (Rhodes, 2011). Within a week, surprisingly, a then-PhD student, Martin O’Leary, came up with an algorithm based on his domain knowledge in glaciology that outperformed the state-of-the-art solution in astronomy.
%Martin, however, was unable to have the last laugh. The final winner was a team of two researchers from the University of California, Irvine. In fact, two out of the top three were teams, while only one attended as an individual. 

%%%%%%%%%%
%%%%%%%%%%
\section{Related Work}
\smallskip
This is where you put related work. Cite a paper~\parencite{zhao2019modelling}.
Also there's a table~\cref{tab:table1}.
\blindtext

\blindtext[3]
%%%%%%%%%%
\bigskip
\section{Research Plan and Methodology}
\smallskip
\blindtext[1]
Refer to a figure like this~\cref{fig:1}.

\blindtext[3]


%%%%%%%%%%%% END MAIN TEXT %%%%%%%%%%%%
%%%%%%%%%%%% fig and tab %%%%%%%%%%%%
\clearpage
\pagenumbering{arabic}
\setcounter{page}{1}
\section{Figures and Tables}
\input{3_Figures}
\input{4_Tables}


%%%%%%%%%%%% REFERENCES %%%%%%%%%%%%
\clearpage
\pagenumbering{arabic}
\setcounter{page}{1}
\section{References}
\printbibliography[heading=none]

\end{document}
